\documentclass[../main.tex]{subfiles} % To be correctly processed by subfiles

% Do not use any packages here, write imports directly to main.tex

\begin{document}

Blind source separation (or blind signal separation) is a classical problem in signal processing that lies in restoring the original sources from multiple mixed observations. The typical example is a "cocktail party problem": a few people talk in the same room simultaneously and we need to extract the speech of every person given multiple recordings from different microphones placed in that room.

Formally, we denote  each observed recording as $\bx_i$ and each latent original signal as $\bs_j$. Then, linear mixing procedure might be written as 

\[ \bx_i = a_{i1} \bs_1 + \ldots + a_{in} \bs_n, i = \overline{1, m}, \]
or, in a matrix form, 
\[ X = A S, \]
where $A = \left[a_{ij} \right]$ is called a mixing matrix and is unknown. Usually, we assume that the number of recordings $m$ equals the number of original sources $n$.

Our problem is then to find an estimate of an "unmixing matrix" $W = A^{-1}$  and restore original sources $S = W X$.
 
\end{document}