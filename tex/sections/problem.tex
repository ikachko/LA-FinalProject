\documentclass[../main.tex]{subfiles} % To be correctly processed by subfiles

% Do not use any packages here, write imports directly to main.tex

\begin{document}

% Basics of DSP
\subsection{Basics of digital sound signal processing}


% What is signal
% What is sound, analogue vs digital, sample rate
% Time domain, frequency domain

Now, when we get the basics of sound signal processing, let's return to the blind source separation problem.

% Blind source separation formal problem statement
\subsection{Blind source separation}

As mentioned earlier, we have $m$ observed recordings of $n$ original sources. Usually, we assume that the number of recordings $m$ equals the number of original sources $n$.

Formally, we represent each observed recording as a discrete signal $\bx_i$ and each source as a latent discrete signal $\bs_j$. Then, linear mixing procedure might be written as 
\[ \bx_i = a_{i1} \bs_1 + \ldots + a_{in} \bs_n, i = \overline{1, n}, \]
or, in a matrix form, 
\[ X = A S, \]
where $A = \left[a_{ij} \right]$ is called a mixing matrix and is unknown. 

The problem is then to find an estimate of the "unmixing matrix" $W = A^{-1}$  and restore original sources $S = W X$. 

Each algorithm has its own way of extracting original signals. PCA attempts to make signals orthogonal, ICA -- as independent as possible. NMF simply performs an approximate factorization with a hope that the matrices will be interpretable. We describe it in more detail in section 3. Except that, we also need a unique domain-specific metrics that will allow us to quantitatively assess the quality of separation independent of algorithm used to perform it.

% Nuances in a musical domain
\subsection{Quality assessment}

% What is timbre?
% Unique characteristics of musical instruments - pitch, power, tone color


% How to quantify timbre?

% How 

\end{document}
