\documentclass[../main.tex]{subfiles} % To be correctly processed by subfiles

% Do not use any packages here, write imports directly to main.tex

\begin{document}

This section explains the chosen solution approaches.

\subsection{ICA}

\textbf{I}ndependent \textbf{C}omponent \textbf{A}nalysis  (shortly \textbf{ICA}) is a statistical model, that assumes statistical \textit{independency} of the components of $A$ from (2). The independent components are latent. It means that they cannot be directly observed. Mixing matrix $S$ from (2) is assumed to be unknown too. Below, we will explain the FastICA algorithm for iterative approximation of $A$ and $S$ matrices.

\subsection{Preprocessing for ICA}

\subsubsection{Centering data}
To simplify algorithm, data must be centered. The easiest way to center \textbf{x} is to substract its mean vector $\textbf{m} = E\{\textbf{x}\}$ to make $\textbf{x}$ a zero-mean variable.

\subsubsection{Whitening}
After centering, another useful preprocessing in ICA is a \textit{whitening} of variables. It's a linear transformation $\mathbf{L}: \mathbb{R}^{N \times M} \to \mathbb{R}^{N \times M}$ that makes component uncorrelated and have variance one.


\begin{algorithm} 
\begin{algorithmic}
\caption{FastICA algorithm}\label{fastica}
\item \textbf{Input}: $C$ Number of desired components
\item \textbf{Input}: $\mathbf{X} \in \mathbb{R}^{N \times M}$ Prewhitened matrix, where each column represents an $N$-dimensional sample, where $C \le N$ 
\item \textbf{Output}: $\mathbf{W} \in \mathbb{R}^{N \times C}$ Un-mixing matrix where each column projects $\mathbf{X}$ onto independent component.
\item \textbf{Output}: $\mathbf{S} \in \mathbb{R}^{C \times M}$ Independent components matrix, with $M$ columns representing a sample with $C$ dimensions.

\Function{FastICA}{$A, C$}
\For{p \textbf{in} 1 to $C$}
	\State{$\mathbf{w}_p \gets $ Random vector of length N}
	\While{$\mathbf{w}_p$ changes}
		\State{$\mathbf{w_p} \gets \frac{1}{M} \mathbf{X}g (\mathbf{w_p}^\top \mathbf{X})^\top - \frac{1}{M} g^{'} (\mathbf{w_p}^\top \mathbf{X}) \mathbf{1} \mathbf{w_p}$}
		\State{$\mathbf{w_p} \gets \mathbf{w_p} - \sum_{j = 1}^{p - 1}(\mathbf{w_p}^\top \mathbf{w_j})\mathbf{w_j}$}
		\State{$\mathbf{w_p} \gets \frac{\mathbf{w_p}}{||\mathbf{w_p}||}$}
	\EndWhile
\EndFor
\State{\textbf{output} \textbf{W} $\gets [\mathbf{w_1}, \dots, \mathbf{w_C}]$ }
\State{\textbf{output} \textbf{S} $\gets \textbf{W}^\top \textbf{X}$}
\EndFunction

\end{algorithmic}
\end{algorithm}



\subsection{NMF}

NMF description.	

\end{document}