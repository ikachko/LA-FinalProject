\documentclass[../main.tex]{subfiles} % To be correctly processed by subfiles

% Do not use any packages here, write imports directly to main.tex

\begin{document}

\subsection{Data}

We used data from an online source Puremix, that provides a set of individualy recorded instruments from a single track. For our experiment we used \textit{Iron Sheik -- Dry Clean Only} track. It contains a 30 unique recordings in a .wav format:
\begin{list}{--}{spacing}
	\item 16 drums
	\item 5 guitars
	\item 9 vocals
\end{list}

\subsection{Input}


We have picked one audio from each class and mixed them in a different combinations.

Let $S$ be a $N \times K$ matrix, where $K$ is a number of independent signals and $N$ be a length of each signal:


$G, V, D --\text{vectors of a guitar, vocal and drums signals}$

$$S = \begin{bmatrix}
G & V & D
\end{bmatrix}$$

Then, we used a $K \times K$ coefficient matrix $H$ that will multiply each instrument by a coefficient and sum them up.

For out experiment we used matrix with such coefficients:

$$H = \begin{bmatrix}
0.2 & 0.6 & 0.4 \\
0.5 & 0.2 & 0.1 \\
0.3 & 0.2 & 0.5
\end{bmatrix}$$

So our resulting matrix of mixed tracks is:

$$X = H S$$

that can be interpreted like this:

$X_i = V * H_{i, 1} + G * H_{i, 2} + D * H_{i, 3} $ -- i-th mixed track


\subsection{Tools}

To compute a numerical experiments we used a \textbf{sklearn.decomposition} module.


For an ICA we used \textbf{sklearn.decomposition.FastICA}

For an NMF we used \textbf{sklearn.decomposition.NMF}









\end{document}