\documentclass[../main.tex]{subfiles} % To be correctly processed by subfiles

% Do not use any packages here, write imports directly to main.tex

\begin{document}

<<<<<<< HEAD
\subsection{Data}

We used data from an online source Puremix, that provides a set of individualy recorded instruments from a single track. For our experiment we used \textit{Iron Sheik -- Dry Clean Only} track. It contains a 30 unique recordings in a .wav format:
\begin{list}{--}{spacing}
	\item 16 drums
	\item 5 guitars
	\item 9 vocals
\end{list}

\subsection{Input}


We have picked one audio from each class and mixed them in a different combinations.

Let $S$ be a $N \times K$ matrix, where $K$ is a number of independent signals and $N$ be a length of each signal:


$G, V, D --\text{vectors of a guitar, vocal and drums signals}$

$$S = \begin{bmatrix}
G & V & D
\end{bmatrix}$$

Then, we used a $K \times K$ coefficient matrix $H$ that will multiply each instrument by a coefficient and sum them up.

For out experiment we used matrix with such coefficients:

$$H = \begin{bmatrix}
0.2 & 0.6 & 0.4 \\
0.5 & 0.2 & 0.1 \\
0.3 & 0.2 & 0.5
\end{bmatrix}$$

So our resulting matrix of mixed tracks is:

$$X = H S$$

that can be interpreted like this:

$X_i = V * H_{i, 1} + G * H_{i, 2} + D * H_{i, 3} $ -- i-th mixed track


\subsection{Tools}

To compute a numerical experiments we used a \textbf{sklearn.decomposition} module.


For an ICA we used \textbf{sklearn.decomposition.FastICA}

For an NMF we used \textbf{sklearn.decomposition.NMF}








=======
With a clearly stated problem and all key methods described, let's apply them to separate a few songs. The procedure is described bellow. All the source code and the data can be found at \href{https://github.com/ikachko/LA-FinalProject}{GitHub repository}.

First, we take a pre-mixed multitrack recording of some song. We used the following ones, which are freely available at mixing enthusiasts forums:

\begin{itemize}[leftmargin=4em]
	\item Dry Clean Only -- Iron Sheik (guitar, vocal, bass and drums tracks);
	\item Oasis -- Wonderwall (guitar, vocal, bass, drums and strings+guitar tracks);
	\item All Star -- Smash Mouth (guitar, vocal, bass, drums, effects tracks).
\end{itemize}

We load each track separately and store it as a column of a matrix $S$ of sources.
It is also useful to build a spectrogram for each track to get a visual sense of it and be able to compare the spectrograms of original and estimated sources. 
To handle records (loading tracks, building spectrograms) we use a package \lstinline{librosa}.

Next, we generate a random mixing matrix $A$ to produce the required number of observations by applying $X = S A$.

NMF requires some further preprocessing, namely, $X$ must be non-negative.
We can achieve this by subtracting from each column of $X$ the smallest value in it, or by applying the min-max normalization to each column. 

Now, we can apply the separation. 
To perform it, we use \lstinline{sklearn.decompose} module which provides \lstinline{FastICA} and \lstinline{NMF} classes with simple interface.

The main questions we set for this experiment are:

\begin{itemize}[leftmargin=4em]
	\item Can any of the methods produce reasonable separations?
	\item Can the methods handle the blended source tracks (e.g., strings+guitar in Wonderwall)?
	\item How audible are the artifacts?
\end{itemize}

The results and conclusions are provided in the next section.
>>>>>>> 08a2876dcfd7ff6025a4f0567f0dd7795468db49

\end{document}