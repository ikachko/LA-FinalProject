\documentclass[a4paper,12pt]{article}

\textheight=245truemm 
\textwidth=175truemm 
\hoffset=-15truemm 
\voffset=-20truemm

\usepackage{mathtools}
\usepackage{amsmath,amsfonts,amssymb,amsthm}

\usepackage{hyperref} % hyperlinks
\usepackage{listings}[language=Python] % code
\usepackage{enumitem} % format lists
\usepackage{indentfirst} % Indent the first paragrapgh
\usepackage{subfiles} % For multi-file projects

\setlist{nosep}

\newcommand{\ba}{{\mathbf a}}
\newcommand{\bb}{{\mathbf b}}
\newcommand{\bc}{{\mathbf c}}
\newcommand{\bd}{{\mathbf d}}
\newcommand{\be}{{\mathbf e}}
\newcommand{\bm}{{\mathbf m}}
\newcommand{\bn}{{\mathbf n}}
\newcommand{\bp}{{\mathbf p}}
\newcommand{\bq}{{\mathbf q}}
\newcommand{\br}{{\mathbf r}}
\newcommand{\bs}{{\mathbf s}}
\newcommand{\bu}{{\mathbf u}}
\newcommand{\bv}{{\mathbf v}}
\newcommand{\bw}{{\mathbf w}}
\newcommand{\bx}{{\mathbf x}}
\newcommand{\by}{{\mathbf y}}
\newcommand{\bz}{{\mathbf z}}



\title{ Blind Source Separation for Musical Recordings }
\author{ Dmitry Lekhovitsky, Illia Kachko }
\date{ }

\begin{document}

\maketitle

\begin{abstract}
	In this project, we consider the problem of blind source separation -- given multiple observations of a linearly mixed signal, we aim to restore the mixing matrix and original sources.
	There are two main methods to perform this problem that heavily rely on linear algebra, namely, Independent Component Analysis (ICA) and Non-negative Matrix Factorization (NMF).
	Particularly, we apply these methods to the problem of separating a musical recording into individual tracks and show that ICA frequently produces the reasonable solution, while NMF fails in most cases.
	The main restriction of the classical approaches is the requirement that the number of recordings must be equal to the number of sources, which is rarely the case. 
	To get the needed number of observations we artificially mix the multi-tracks of a few songs with random mixing matrices.
	In our future work, we plan to remove this constrain by applying deep learning methods, and also introduce the formal quality assessment metric.
\end{abstract}


\section{Introduction}
\subfile{sections/introduction}

\section{Problem statement}
\subfile{sections/problem}

\section{Solution approaches}
\subfile{sections/methods}

\section{Numerical experiment}
\subfile{sections/experiment}

\section{Results}
\subfile{sections/results}

% \begin{bibliography}

% \bibitem{ica}
	
% \end{bibliography}

\end{document}